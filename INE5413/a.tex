\documentclass{beamer}
\usetheme{Boadilla}

\usepackage[brazil]{babel}
\usepackage[utf8]{inputenc} %Codificação de entrada
\usepackage[T1]{fontenc} %Codificação do PDF gerado
\usepackage{amsmath, amsfonts, amsthm}
\usepackage{textcomp} %Companion fonts

\theoremstyle{definition}
\newtheorem*{definicao}{Definição}

\usepackage{tikz}
\usetikzlibrary{positioning} % above = of
\usetikzlibrary{calc} % Vetores ($ $)
\usetikzlibrary{fit}

\tikzset{
    node distance = 2cm,
    line width = 3pt,
    vertice/.style = {
        shape = circle,
        inner sep = 3pt,
        fill = #1
    },
    vertice/.default = black,
    flor/.style = {
        shape = circle,
        inner sep = 5pt,
        fill = #1
    },
    flor/.default = green,
    emparelhamento/.style = {
        color = blue,
        line width = 1.2pt
    },
    aresta/.style = {
        color = #1,
        line width = 1.2pt
    },
    aresta/.default = black,
    caminho/.style = {
        color = #1,
        line width = 5pt
    },
    caminho/.default = yellow
}

\newcommand{\GrafoExemploXOR}{ % Este comando deve estar dentro de um tikzpicture
    %     A-----E=====F
    %   / ||    |
    %  B  ||    |  As arestas emparelhadas estão desenhadas mais grossas
    %   \ ||    |
    %     C-----D    

    % Anatomia deste comando node:
    % Posição: (0, 0)  -- também pode ser uma coordenada
    % Operação especial em caminho: node -- Adicionada ao final do desenho. Ignora draw/fill/filldraw/etc.
    % Atributos da operação especial:
    %   circle - forma
    %   inner sep - especifica a separação entre a borda do círculo e o texto.
    %   fill - ordena ao TikZ desenhar e preencher o círculo na tela de azul
    % Nome do nodo: Nodo -- este nome pode ser usado na posição também
    % Label: {A} -- pode ser vazia
    %\path (0, 0) node [shape=circle, inner sep = 2pt, fill=blue] (A) {};

    \coordinate (Centro); %Centro do triângulo

    \path (Centro) arc [start angle = 0, end angle = 60, radius = 1cm] % Agora, estamos no lugar do nodo do canto superior esquerdo.
        coordinate (A) %Vamos botar uma coordenada aqui.
        arc [start angle = 60, end angle = 180, radius = 1cm] % O comando node não adiciona um ponto à curva, então continuamos
        coordinate (B)                                        % com a referência ao final do arco anterior
        arc [start angle = 180, end angle = 300, radius = 1cm]
        coordinate (C);

    %\path node[vertice=green] (D) at ($ (C) + (3, 0) $) {};
    % Esta coordenada é uma combinação linear. É delimitada por ($ e $); a coordenada será a soma de (C) com (3, 0).
     
    \path let \p1 = ($ (A) - (C) $) in %No registrador p1 (ponto 1), armazene a coordenada ($ (C) - (A) $).
        coordinate (D) at ($ (C) + (\y1, 0) $) {} %\y1 é a coordenada y do registrador p1.
        coordinate (E) at ($ (A) + (\y1, 0) $) {}
        coordinate (F) at ($ (A) + 2*(\y1, 0) $) {};
}
\newcommand{\ArestasExemploXOR}{
    \draw [aresta] (A) -- (B) -- (C) -- (D) -- (E) -- (A);
    \draw [emparelhamento] (A) -- (C) (E) -- (F);
    \node [shape = circle, inner sep = 3pt, fill = blue] at (A) {};
    \node [vertice] at (B) {};
    \node [vertice = blue] at (C) {}; %Este nodo e o primeiro possuem exatamente as mesmas configurações.
    \node [vertice] at (D) {};
    \node [vertice = blue] at (E) {};
    \node [vertice = blue] at (F) {};
}

\begin{document}

\title{Caminhos, Árvores e Flores}
\subtitle{Um algoritmo polinomial para emparelhamento}
\author[Feitosa, Royer, Gascho]{Igor Henrique Grajefe Feitosa \\ Tiago Royer \\ Wagner Fernando Gascho}
\institute[UFSC]{Universidade Federal de Santa Catarina}
\frame{\titlepage}

\begin{frame}
    \frametitle{Definições}

    Um emparelhamento de um grafo é um subconjunto de arestas que não se tocam.

    \begin{figure}[h]
        \begin{tikzpicture}
            \GrafoExemploXOR

            \ArestasExemploXOR
        \end{tikzpicture}
    \end{figure}

    Vértices que são tocados por arestas do emparelhamento estão \emph{emparelhados}; os demais estão \emph{expostos}.
\end{frame}

\begin{frame}
    Dado um emparelhamento M:
    \begin{itemize}
        \item Um caminho é \emph{M-alternado} se suas arestas se alternam entre estar fora e dentro do emparelhamento.
            \begin{figure}[h]
                \begin{tikzpicture}
                    \GrafoExemploXOR
                    \draw[caminho]  (B) -- (C) -- (A) -- (E) -- (F);
                    \ArestasExemploXOR
                \end{tikzpicture}
            \end{figure}
        \pause
        \item Um caminho M-alternado é \emph{M-aumentante} se ele ligar dois vértices expostos.
            \begin{figure}[h]
                \begin{tikzpicture}
                    \GrafoExemploXOR
                    \draw[caminho]  (B) -- (A) -- (C) -- (D);
                    \ArestasExemploXOR
                \end{tikzpicture}
            \end{figure}
    \end{itemize}
\end{frame}

\begin{frame}
    \frametitle{Diferença simétrica -- Ou exclusivo}
    Podemos fazer um ou-exclusivo (XOR) entre um caminho M-aumentante e o emparelhamento M para aumentar sua cardinalidade:
    \begin{figure}
        \begin{tikzpicture}
            \GrafoExemploXOR
            \draw[caminho]  (B) -- (A) -- (C) -- (D);
            \ArestasExemploXOR
        \end{tikzpicture}

        \begin{tikzpicture}
            \GrafoExemploXOR
            \draw [aresta] (B) -- (C) -- (A) -- (E) -- (D);
            \draw [emparelhamento] (A) -- (B) (C) -- (D) (E) -- (F);
            \node [vertice = blue] at (A) {};
            \node [vertice = blue] at (B) {};
            \node [vertice = blue] at (C) {};
            \node [vertice = blue] at (D) {};
            \node [vertice = blue] at (E) {};
            \node [vertice = blue] at (F) {};
        \end{tikzpicture}
    \end{figure}

    O algoritmo de Hopcroft-Karp encontra emparelhamento máximo em grafos bipartidos procurando por caminhos M-aumentantes.
\end{frame}

\begin{frame}
    \frametitle{Execução: busca em largura}
    \onslide<1>{Enquanto houverem vértices não-emparelhados, busque um caminho M-aumentante.}
    \begin{figure}
        \begin{tikzpicture}
            % A --- B
            %||     | \
            %||     |  E
            %||     |// \
            % C --- D    H
            %  \   /  \ /
            %    F === G
            \coordinate (A);
            \coordinate [right = of A] (B);
            \coordinate [below = of A] (C);
            \coordinate [right = of C] (D);
            \coordinate (E) at ($(B)!1!60:(D)$); % interpole (B) com [(D) rotacionado 60 graus sobre (B)] com 1 (=100%) de preferencia para (D).
            \coordinate (F) at ($(C)!1!-60:(D)$); 
            \coordinate [right = of F] (G);
            \coordinate (H) at ($(G)!1!-90:(D)$); 

            \onslide<1>{
                \node [above right of = H, align = center] (Texto) {Começaremos\\aqui};
                \draw[->, shorten >=5pt, line width = 2pt] (Texto) -- (H);
            }

            \onslide<2->{
                \draw [caminho] (H) -- (E) (H) -- (G);
            }
            \onslide<3->{
                \draw [caminho] (E) -- (D) (G) -- (F);
            }
            \onslide<4->{
                \draw [caminho] (G) -- (D) -- (C) -- (F) -- (D) -- (B);
            }
            \onslide<5->{
                \draw [caminho = red] (H) -- (E) -- (D) -- (B);
            }

            \draw [aresta] (A) -- (B) -- (D) -- (C) -- (F) -- (D) -- (G) -- (H) -- (E) -- (B);
            \draw [emparelhamento] (A) -- (C) (E) -- (D) (F) -- (G);
            \foreach \posicao in {A, C, F, G, D, E} {
                \node [vertice = blue] at (\posicao) {};
            }
            \foreach \posicao in {B, H} {
                \node [vertice] at (\posicao) {};
            }
        \end{tikzpicture}

        \onslide<6->{Este procedimento basta para grafos bipartidos --- entretanto...}
    \end{figure}
\end{frame}

\begin{frame}
    \frametitle{Contra-exemplo para grafos quaisquer}
    Ramo de busca possível:

    \begin{figure}
        \begin{tikzpicture}
            %       A    B --- C
            %       |  // |  //
            %       | //  | //
            % F --- E ---- D
            \coordinate (A);
            \coordinate [right = of A] (B);
            \coordinate [right = of B] (C);
            \coordinate [below = of B] (D);
            \coordinate [left = of D] (E);
            \coordinate [left = of E] (F);

            \onslide<2-> { \draw[caminho] (A) -- (E); }
            \onslide<3-> { \draw[caminho] (E) -- (B); }
            \onslide<4-> { \draw[caminho] (B) -- (D); }
            \onslide<5-> { \draw[caminho] (D) -- (C); }
            \onslide<6-> { \draw[caminho] (C) -- (B); }
            \onslide<7-> { \draw[caminho = red] (B) -- (E); }
            \onslide<8-> { \draw[caminho = red] (E) -- (F); }

            \draw[aresta] (A) -- (E) -- (F) (E) -- (D) -- (B) -- (C);
            \draw[emparelhamento] (E) -- (B) (C) -- (D);

            \foreach \posicao in {B, C, D, E} \node [vertice = blue] at (\posicao) {};
            \foreach \posicao in {A, F} \node [vertice] at (\posicao) {};
        \end{tikzpicture}
    \end{figure}

    \onslide<9->{Problema: o ``caminho M-aumentante'' encontrado possui auto-interseção.}
\end{frame}

\begin{frame}
    \frametitle{Solução: Flores!}
    \begin{definicao}
        Um ciclo de tamanho $2k+1$ é dito ser uma \emph{flor} se $k$ arestas deste ciclo estão no emparelhamento considerado
        e o único vértice que não é tocado por estas $k$ arestas é conectado por um caminho M-alternado a um vértice exposto.
    \end{definicao}

    \begin{figure}
        \begin{tikzpicture}[node distance = 1.2cm]
            % ABCDE fazem parte da flor, F é emparelhado com A e G é exposto.
            % G -- F == (ABCDE)

            \coordinate (Centro) at (0, 0); % Centro da flor
            \foreach \nome/\rotacao in {A/0, B/1, C/2, D/3, E/4}
                \path (Centro) ++(-1.2cm, 0) arc (180 : 180 + \rotacao * 72 : 1.2cm) coordinate (\nome);
            \coordinate [left = of A] (F);
            \coordinate [left = of F] (G);
            \filldraw[fill = green!50, draw = green!20!black] (0, 0) circle [radius = 1.5cm];

            \draw[aresta] (E) -- (A) -- (B) (C) -- (D) (F) -- (G);
            \draw[emparelhamento] (B) -- (C) (D) -- (E) (A) -- (F);
            \foreach \posicao in {A, B, C, D, E, F} \node [vertice = blue] at (\posicao) {};
            \node [vertice] at (G) {};
        \end{tikzpicture}
    \end{figure}
\end{frame}

\begin{frame}
    \frametitle{Contração de flores}
    Podemos contrair as flores num único nodo:
    \begin{figure}
        \begin{tikzpicture}[node distance = 1.2cm]
            \coordinate (Centro) at (0, 0); % Centro da flor
            \foreach \nome/\rotacao in {A/0, B/1, C/2, D/3, E/4}
                \path (Centro) ++(-1.2cm, 0) arc (180 : 180 + \rotacao * 72 : 1.2cm) coordinate (\nome);
            \coordinate [left = of A] (F);
            \coordinate [left = of F] (G);
            \filldraw[fill = green!50, draw = green!20!black] (0, 0) circle [radius = 1.5cm];

            \draw[aresta] (E) -- (A) -- (B) (C) -- (D) (F) -- (G);
            \draw[emparelhamento] (B) -- (C) (D) -- (E) (A) -- (F);
            \foreach \posicao in {A, B, C, D, E, F} \node [vertice = blue] at (\posicao) {};
            \node [vertice] at (G) {};

            % X --- Y === Z (Z é uma flor)
            %\coordinate[below = of Centro, node distance = 5cm] (Z);
            \path (Centro) -- ++(0, -3cm) coordinate (Z);
            \coordinate[left = of Z] (Y);
            \coordinate[left = of Y] (X);

            \draw[aresta] (X) -- (Y);
            \draw[emparelhamento] (Y) -- (Z);
            \node[vertice = blue] at (Y) {};
            \node[flor] at (Z) {};
            \node[vertice] at (X) {};

            \path (Centro) -- (Z) node[pos=0.6] (S) {} node[pos=0.8] (T) {};
            \draw[->, color = red, line width = 6pt, >=stealth] (S) -> (T);
        \end{tikzpicture}
    \end{figure}
\end{frame}

\begin{frame}
    Recuperar caminho M-aumentante
    \begin{figure}
        \begin{tikzpicture}
        \onslide<2->{
            \coordinate (Centro) at (0, 0); % Centro da flor
            \foreach \nome/\rotacao in {A/0, B/1, C/2, D/3, E/4}
                \path (Centro) ++(-1.2cm, 0) arc (180 : 180 + \rotacao * 72 : 1.2cm) coordinate (\nome);
            \coordinate [left = of A] (F);
            \coordinate [left = of F] (G);
            \coordinate [right = of D] (H);
            \filldraw[fill = green!50, draw = green!20!black] (0, 0) circle [radius = 1.5cm];

            \draw[caminho] (G) -- (F) -- (A) -- (E) -- (D) -- (H);
            \draw[emparelhamento] (B) -- (C) (D) -- (E) (A) -- (F);
            \draw[aresta] (E) -- (A) -- (B) (C) -- (D) -- (H) (F) -- (G);
            \foreach \posicao in {A, B, C, D, E, F} \node [vertice = blue] at (\posicao) {};
            \node [vertice] at (G) {};
            \node [vertice] at (H) {};
        }

            % X --- Y === Z --- W (Z é uma flor)
            %\coordinate[below = of Centro, node distance = 5cm] (Z);
            \path (Centro) -- ++(0, +3cm) coordinate (Z);
            \coordinate[left = of Z] (Y);
            \coordinate[left = of Y] (X);
            \coordinate[right = of Z] (W);

            \draw[caminho] (X) -- (Y) -- (Z) -- (W);
            \draw[emparelhamento] (Y) -- (Z);
            \draw[aresta] (X) -- (Y) (Z) -- (W);
            \node[vertice = blue] at (Y) {};
            \node[flor] at (Z) {};
            \node[vertice] at (X) {};
            \node[vertice] at (W) {};

        \onslide<2->{
            \path (Centro) -- (Z) node[pos=0.6] (S) {} node[pos=0.8] (T) {};
            \draw[->, color = red, line width = 6pt, >=stealth] (T) -> (S);
        }
        \end{tikzpicture}
    \end{figure}
\end{frame}

\begin{frame}
    \frametitle{De volta ao contra-exemplo}
    Encontramos uma flor:

    \begin{figure}
        \begin{tikzpicture}
            %       A    B --- C
            %       |  // |  //
            %       | //  | //
            % F --- E ---- D
            \coordinate (A);
            \coordinate [right = of A] (B);
            \coordinate [below = of A] (E);
            \coordinate [left = of E] (F);

            \onslide<1>{
                \coordinate [right = of B] (C);
                \coordinate [below = of B] (D);
            }

                \onslide<1>{
                \node[fill = green!50, draw = green!20!black, fit = (B) (C) (D), inner sep = 0.2cm, shape = circle ] {} ;
            }

            \draw[aresta] (A) -- (E) -- (F);
            \draw[emparelhamento] (E) -- (B);

                \onslide<1> { 
                    \draw[aresta] (E) -- (D) -- (B) -- (C);
                    \draw[emparelhamento] (C) -- (D);
                    \foreach \posicao in {C, D} \node [vertice = blue] at (\posicao) {};
                }

            \foreach \posicao in {B, E} \node [vertice = blue] at (\posicao) {};
            \foreach \posicao in {A, F} \node [vertice] at (\posicao) {};
            \onslide<2-> { \node [flor] at (B) {}; }
        \end{tikzpicture}

        \onslide<3->{A busca em largura pode ser modificada para detectar flores enquanto procura por caminhos aumentantes.}
    \end{figure}
\end{frame}

\begin{frame}
    \frametitle{Algoritmo}
    \begin{itemize}
        \item Enquanto o emparelhamento não for máximo
            \begin{itemize}
                \item Para cada vértice exposto, faça uma busca em largura:
                    \begin{enumerate}
                        \item Se encontrar uma flor, contraia.
                        \item Se encontrar um caminho, faça um ou-exclusivo com o emparelhamento atual e volte para o começo.
                        \item Se encontrar apenas becos sem saída, desista: este nodo não pode ser emparelhado.
                    \end{enumerate}
            \end{itemize}
    \end{itemize}
\end{frame}
\end{document}
